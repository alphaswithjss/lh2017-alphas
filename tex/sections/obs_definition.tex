
In this section we briefly review the definitions of the observables and grooming procedures that we will focus on in this paper.


%%%%%%%%%%%%%%%%%%%%%%%%%%%%%%%%%%%%%%%
\subsection{Observables}\label{sec:shape_def}
%%%%%%%%%%%%%%%%%%%%%%%%%%%%%%%%%%%%%%%


The simplest class of jet shape observables that have sensitivity to the value of $\alpha_s$ are two-point correlation functions. There has been significant theoretical study of these objects, and their perturbative behavior is understood to relatively high accuracy. Here we give definitions for both jets in $e^+e^-$ as well as in $pp$.

\ijm{copied in, need to rewrite a bit}

For a set of particles $\{i\}$, in a jet $J$, the two-point correlation functions in $e^+e^-$ collisions are defined as
\begin{align}\label{eq:eee2}
\left.\ecf{2}{\alpha}\right|_{e^+e^-}&=\frac{1}{E_J^2}\sum_{i<j\in J} E_i E_j\left(
\frac{2p_i\cdot p_j}{E_i E_j}
\right)^{\alpha/2}\,, 
\end{align}
\gs{I am not sure this is relevant here but the definition we used for
  the LH q/g study in 2015 had $(\theta_{ij}/R)^\alpha$, which is
  quite different (different normalisation, differnt at large angle,
  different with non-zero masses). But we're not using it here, so
  can't we just drop this and focus on the $pp$ definition?}   For
jets produced in $pp$ collisions, the two-point correlation functions
are simply modified as
\begin{align}\label{eq:ppe2}
\left.\ecf{2}{\alpha}\right|_{pp}&=\frac{1}{p_{TJ}^2}\sum_{i<j\in J} p_{Ti} p_{Tj}R_{ij}^\alpha\,, 
\end{align}
\gs{The definition in the rivet-analysis folder as well as the one
  I've used in my resummation ocde includes a $1/R^\alpha$
  normalisation.}  Here $R_{ij}$ is the distance between particles $i$
and $j$ in the pseudorapidity-azimuth \gs{We're definitely using
  rapidities, not pseudo-rapidities} angle plane.  For jets that are
central ($p_{TJ} \sim E_J$) and if all emissions in the jet are
collinear, the definitions of the two- and three-point energy
correlation functions for jets in $pp$ collisions are equivalent to
those for $e^+e^-$ collisions.

The angular exponent $\alpha$ in the definition of the two-point correlation functions is a parameter that controls sensitivity to wide-angle emissions.  For jets that consist of massless particles, the two-point energy correlation function in $e^+e^-$ collisions reduces to a function of the jet mass, $m_J$, if $\alpha=2$:
\begin{equation}
\left.\ecf{2}{2}\right|_{e^+e^-} = \frac{m_J^2}{E_J^2}\,.
\end{equation} 

Throughout this paper, we will primarily restrict ourselves to the
case of $\alpha=2$. This is the case for which the theory
understanding is best, and which has been perturbatively calculated to
NNLL~\cite{Frye:2016okc,Frye:2016aiz}. \gs{I think the theory
  understanding is valid for any $\alpha$ (the best way to prove this
  is that the code resummation code works for any $\alpha$). Note:
  we've only provided the mass to CMS and ATLAS but it is simply
  because that is what is measured.} It is, however, and interesting
question as to whether other values of $\alpha$ could provide improved
sensitivity to $\alpha_s$. Another benchmark value is $\alpha=1$,
which corresponds to $k_T$ (or broadening) instead of mass.

%a broadening or $k_T$ type measure.

%%%%%%%%%%%%%%%%%%%%%%%%%%%%%%%%%%%%%%%
\subsection{Grooming Techniques}\label{sec:groom_tech}
%%%%%%%%%%%%%%%%%%%%%%%%%%%%%%%%%%%%%%%

Starting from a jet identified with an IRC safe jet algorithm (such as
anti-$k_t$ \cite{Cacciari:2008gp}), the soft drop algorithm is defined
using Cambridge/Aachen (C/A) reclustering
\cite{Dokshitzer:1997in,Wobisch:1998wt,Wobisch:2000dk}.  Specializing
to the case of $\beta=0$, where Soft Drop is equivalent to Modified
Mass Drop Tagger~\cite{Dasgupta:2013ihk}, the algorithm proceeds as
follows: \gs{The descriptino below is valid for any $\beta$, so I'd
  just drop the $\beta=0$ part here and mention, below the description
  ``This generalises the modified Mass Drop
  Tagger~\cite{Dasgupta:2013ihk} which corresonds to $\beta=0$ in the
  Soft Drop procedure described above.''}
\begin{enumerate}

\item Recluster the jet using the C/A clustering algorithm, producing an angular-ordered branching history for the jet.

\item Step through the branching history of the reclustered jet.  At each step, check the soft drop condition
\begin{align}\label{eq:sd_cut}
\frac{\min\left[ p_{Ti}, p_{Tj}  \right]}{p_{Ti}+p_{Tj}}> \zcut \left(   \frac{R_{ij}}{R}\right)^\beta \,.
\end{align}
Here, $\zcut$ is a parameter defining the scale below which soft radiation is removed.  If the soft drop condition is not satisfied, then the softer of the two branches is removed from the jet.  This process is then iterated on the harder branch.

\item The soft drop procedure terminates once the soft drop condition is satisfied.

\end{enumerate}

For $\beta=0$ non-global logarithms are removed from the distribution.  
\gs{This is a bit misleading: NGLs are also formally removed from Soft
  Drop with $\beta>0$}




