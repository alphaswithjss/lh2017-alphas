%\info{Ben, Ian, Jesse, Grigorios, Maria}

Within the context of Quantum Chromodynamics (QCD), the strong coupling constant ($\alpha_s$) is responsible for the strength of interactions between quarks and gluons.  Governed by this fundamental parameter, a plethora of strong force phenomena emerge to create massive hadrons responsible for most of the visible energy-density in the universe and to create collimated sprays of hadrons known as jets that are ubiquitous at high energy particle colliders.  The internal structure of jets (jet substructure) has been extensively exploited to search for new particles at the Large Hadron Collider (LHC).  Now that both the experimental and theoretical tools of jet substructure are reaching a high level of maturity~\cite{Abdesselam:2010pt,Altheimer:2012mn,Altheimer:2013yza,Adams:2015hiv,Larkoski:2017jix}, it is time to ask if the radiation pattern inside jets can be used to extract fundamental parameters of the SM, such as $\alpha_s$.


Within the context of Quantum Chromodynamics (QCD), the strong
coupling constant ($\alpha_s$) is responsible for the strength of
interactions between quarks and gluons.  Governed by this fundamental
parameter, a plethora of strong force phenomena emerge to create
massive hadrons responsible for most of the visible energy-density in
the universe and to create collimated sprays of hadrons known as jets
that are ubiquitous at high energy particle colliders.  The
uncertainty in $\alpha_s$ is a limiting factor in our predictions for
the stability of the universe~\cite{Andreassen:2017rzq} and the
uncertainty in $\alpha_s$ at a variety of scales sets our
model-independent sensitivity to strongly interacting particles beyond
the Standard Model~\cite{Kaplan:2008pt,Becciolini:2014lya}.  Now that
many scattering processes have been calculated to a high perturbative
order, the uncertainty on $\alpha_s$, $\sigma_{\alpha_s}$, can be
limiting in the overall accuracy e.g. numerically,
$\alpha_s^3\sim \sigma_{\alpha_s}$ so higher order terms can be
smaller than the leading order correction uncertainty (see
e.g. $gg\rightarrow H$ at N$^3$LO).  Determinations of $\alpha_s$
probe a wide variety of physical phenomena and the consistency between
methods is a crucial test of the theory and is important for accurate
predictions at the Large Hadron Collider (LHC) and beyond.


There is a strong motivation for measuring $\alpha_s$.  The uncertainty in $\alpha_s$ is a limiting factor in our predictions for the stability of the universe~\cite{Andreassen:2017rzq} and the uncertainty in $\alpha_s$ at a variety of scales sets our model-independent sensitivity to strongly interacting particles beyond the Standard Model~\cite{Kaplan:2008pt,Becciolini:2014lya}.   Now that many scattering processes have been calculated to a high perturbative order, the uncertainty on $\alpha_s$ can be limiting in the overall accuracy e.g. numerically, $\alpha_s^3\sim \sigma_{\alpha_s}$ so higher order terms can be smaller than the leading order correction uncertainty (see e.g. $gg\rightarrow H$ at N$^3$LO~\cite{Anastasiou:2015ema}).  Determinations of $\alpha_s$ probe a wide variety of physical phenomena and the consistency between methods is a crucial test of the theory and is important for accurate predictions at the Large Hadron Collider (LHC) and beyond.  


% for the inclusive Higgs cross section at N$^3$LO \cite{Anastasiou:2015ema,Anastasiou:2016cez} \ijm{expalin that this means error is like the size of the pert corrections})

The world-average value for $\alpha_s$ at the $Z$ boson mass\footnote{Unless otherwise specified, $\alpha_s$ is always reported at the $Z$ boson mass. } ($m_Z$) is $0.118\pm 0.0013$, a $1.1\%$ total uncertainty~\cite{Olive:2016xmw}.  There has been significant discussion in the community about the challenges and validity of various methods to extract $\alpha_s$, see for instance Refs.~\cite{Bethke:2011tr,Pich:2013sqa,Moch:2014tta,dEnterria:2015kmd,Olive:2016xmw,Salam:2017qdl,Altarelli:2013bpa}.  The most precise (and dominant) input to the world-average is the $\alpha_s$ value from lattice QCD calculations combined with measurements of $B$-hadron mass differences, with an uncertainty that is less than $1\%$.   After the lattice, the most precise determination is from measurements and calculations of thrust and the $C$-parameter in $e^+e^-$~\cite{Abbate:2010xh,Hoang:2015hka,Heister:2003aj,Abdallah:2004xe,Abreu:1996mk,Abreu:1999rc,Biebel:1999zt,Adeva:1992gv,Abbiendi:2004qz,Abe:1994mf}.   These methods are sensitive to very different regimes of QCD and interestingly significantly differ with each other beyond $\sim 3\sigma$, at about the 5\% level.  

%>>> (0.1123-0.1184)/(0.0006**2+0.0015**2)**0.5
%-3.7758052096000596

%https://arxiv.org/pdf/1512.05194.pdf
%http://pdg.lbl.gov/2017/reviews/rpp2017-rev-qcd.pdf

Extractions based on $e^+e^-$ event shapes are sensitive to soft and
collinear regions of phase space, which are modeled with precision
using higher order resummation.  Figure~\ref{fig:propaganda} shows
various values of $\alpha_s$ extracted with next-to-next-to-leading
order (NNLO) calculations \gs{I guess that the thrust case includes a
  resummation. We need to mention the resummation accuracy here
  (probably N$^3$LL).} as well as those used in various Parton Shower
(PS) Monte Carlo (MC) programs\footnote{Note that these do much more
  than the parton shower.  Photons are also included but the impact
  here is quite negligible for electroweak radiation (as opposed to
  $\pi^0\rightarrow\gamma\gamma$)}.  The value of $\alpha_s$ in the
final state PS is also sensitive to the soft and collinear regime of
QCD (albeit in different ways); interestingly, the values used in the
\pythia program~\cite{Sjostrand:2006za,Sjostrand:2007gs} which are fit
to data suggest a higher value of $\alpha_s$ than the lattice result
by about 15\%.  Another challenge with the event shapes extraction is
that the non-perturbative corrections are nearly degenerate with
$\alpha_s$.  This is in part because techniques to parametrically
separate non-perturbative effects from perturbative effects (grooming)
were not mature at the time of the Large Electron Positron collider
(LEP) and the energy scale at LEP did not allow for a large lever-arm
between the energy scales.  The sensitivity to low energy scales where
QCD is non-perturbative is also a key challenge of the lattice
determination.  It would therefore be timely for a precision
extraction of $\alpha_s$ using jets at the LHC.

\begin{figure}
\begin{center}
\includegraphics[width = 0.6\columnwidth]{figures/alphas_propaganda.pdf}
\end{center}
\caption{Various values of $\alpha_s$, including the world-average
  shown as a black line with a grey band~\cite{Olive:2016xmw}.  The
  point labeled \textit{Thrust} is from LEP data and the measurement
  using the CMS top cross-section measurement is the first NNLO
  extraction at the LHC.  The other points are the $\alpha_s$
  parameter value used in the final state shower in various Monte
  Carlo programs.  \herwig and \sherpa use the world-average by
  default while the default or A14 tunes of \pythia employ a higher
  value.  The order at which $\alpha_s$ is utilized is not the same in
  all cases and is given in the $\overline{\text{MS}}$
  scheme. \gs{VAR2 not so clear in the Pythia label.}}
\label{fig:propaganda}
\end{figure}

So far at the LHC, jets have been used mostly as proxies for quarks and gluon four-vectors.  As such, the multiplicity and kinematics of jets in purely hadronic final states can be predicted to high order in $\alpha_s$ in perturbation theory with e.g. NNLOJET~\cite{Currie:2016bfm,Currie:2017ctp} and NLOJet++~\cite{Nagy:2001fj,Nagy:2003tz}.  As a result, measurements of jet multiplicities, energies, and angles can be used to extract $\alpha_s$~\cite{ATLAS:2015yaa,Aaboud:2017fml,Khachatryan:2014waa,CMS:2014mna,Chatrchyan:2013txa}.   Even though these measurements have achieved an uncertainty of $\sim 5\%$, they have not yet been included in the Particle Data Group (PDG) combination~\cite{Olive:2016xmw} as they are using only next-to-leading-order (NLO) theory calculations.  With the recent progress in higher order calculations, this will likely change soon - an extraction using the recent NNLO $t\bar{t}$ cross-section is now included (see Fig.~\ref{fig:propaganda}).

In these extractions, the collinear region defining the internal
structure of jets is ignored due to the lack of precision calculations
\gs{I'd rather say ``irrelevant'' (because of the collinear-safety of
  the jet definition, or because of the finite radius of the jet)
  instead of ``ignored''? Also, we could mention the work on resumming
  $\log(1/R)$ which is relevant in that context (arXiv:1411.5182 and
  arXiv:1602.01110).}.  Jets are not in one-to-one correspondence with
quarks and gluons and their structure is also governed by $\alpha_s$.
Recent experimental and theoretical advances in jet substructure have
shown that the radiation pattern inside jets has great physics
potential~\cite{Abdesselam:2010pt,Altheimer:2012mn,Altheimer:2013yza,Adams:2015hiv,Larkoski:2017jix}.
The focus of jet substructure has been mostly on tagging the origin of
jets and searching for physics beyond the Standard Model.  However,
present and future precision may be sufficient to make a useful
measurement of $\alpha_s$.  Jet grooming - tools for systematically
removing soft and wide-angle radiation - can parametrically separate
the non-perturbative radiation in the jet from the hard
perturbatively-described components.  At a hadron collider, grooming
also mitigates the contribution from the underlying event and
additional nearly simultaneous interactions (pileup).  Theory
calculations for groomed observables have been performed at
NLL~\cite{Marzani:2017kqd,Marzani:2017mva} and
NNLL~\cite{Frye:2016aiz,Frye:2016okc} and will be extended to higher
orders as calculations become available.  Recently,
ATLAS~\cite{Aaboud:2017qwh} and CMS~\cite{CMS-PAS-SMP-16-010} have
demonstrated 5-10\% measurement uncertainties of these calculated
quantities using existing technologies.  The purpose of this note is
to study the feasibility of a measurement of $\alpha_s$ using jet
substructure at the LHC\footnote{Similar techniques would also be
  interesting at both a low and high energy $e^+e^-$ collider; we
  focus here on $pp$ since high quality data are now pouring out of
  the LHC.}.  This is part of a broader program to apply jet grooming
techniques for precision QCD (see also Ref.~\cite{Hoang:2017kmk} for
the top quark mass).  There are significant experimental and
theoretical challenges to achieve success in this program, but with
community synergy we believe it is possible, and furthermore,
represents a concrete goal to push the accuracy and understanding of
jet substructure calculations.



This work is organized as follows. In \Sec{sec:definitions} we define the observables and grooming strategies that will be considered in this paper. In section~\ref{sec:softcomplications} highlight a number of theoretical and experimental benefits of jet grooming, which make groomed observables particularly interesting for extractions of $\alpha_s$. In Sec.~\ref{sec:jetmass} we discuss the sensitivity of groomed mass to the value of $\alpha_s$ both using analytic expressions and Monte Carlo parton shower simulations.   In \Sec{sec:ben_study} an idealized setup is used to illustrate how an extraction of $\alpha_s$ might work at the LHC, using realistic estimates of both the theoretical and experimental uncertainties.  We conclude in Sec.~\ref{sec:future} and discuss future directions for improving both the theoretical and experimental uncertainties for $\alpha_s$ extractions from jet substructure.

\begin{comment}
\begin{itemize}
\item Laying the groundwork for high precision ($\mathcal{O}(1\%)$) extractions of $\alpha_s$ for the full LHC dataset or a future $e^+e^-$ machine.
\item Competitive measurements with existing LHC extractions of $\alpha_s$ (5\%)
\item Probing the tension between thrust (and friends) extractions with lattice (10\%)
\item Parton shower MC (15\%)
\end{itemize}
\end{comment}


%Grooming in general (broader appeal; also try to avoid the use of the word soft drop).

%Then in pp we have to use jets, so that means jet substructure.  Grooming also removes UE, etc.  Also complementary to other pp extraction methods.

%Benefit of hadron collider: larger lever arm (interesting in and of itself, and also to stay away from NP effects).  Also, we have a hadron collider now with precision!

%Ref.~\cite{Hoang:2017kmk}

%Goal of this study: in addition to alpha s itself, we want to demonstrate the broader applicability of grooming techniques for precision QCD; in contrast, most applications so far are for tagging.  Make a connection to the top mass.  Therefore, we begin with a general introduction to these grooming techniques blah blah and then blah blah.

%(mention our simultaneous q/g fraction extraction - this is new independent of JSS)

\begin{comment}
=====

A new class of jet substructure observables have been introduced that are amenable to precision calculations and therefore introduce the possibility of an $\alpha_s$ extraction from the radiation pattern within a jet for the first time.  Precision jet substructure is an active field of research with a general understanding of infrared and collinear (IRC) safe observables at leading logarithm (LL) as well as analytic control over a variety of many other infrared~\cite{Larkoski:2013paa,Larkoski:2015lea} or collinear~\cite{Elder:2017bkd,Krohn:2012fg,Waalewijn:2012sv,Chang:2013rca} unsafe observables.  Many results have also been extended to next-to-leading logarithm (NLL), including the jet mass~\cite{} and other jet shapes~\cite{}.  However, until recently most of these NLL results ignored the contribution from \textit{non-global logarithms} (NGLs).  These contributions from radiation that connects the intra- and inter-jet regions formally arise at NLL and are difficult to calculate in perturbation theory.  By studying the analytic structure of various jet grooming techniques, it was observed that one could remove the sensitivity to NGLs through a modification of existing grooming procedures~\cite{Dasgupta:2013ihk,Dasgupta:2013via}.  Since that time, an entire class of grooming techniques have been constructed with this property, known as \textit{soft drop}~\cite{Larkoski:2014wba}.  --> Grooming in general.

Paragraph that introduces soft drop and the recent calculations~\cite{Frye:2016okc,Frye:2016aiz}, \cite{Marzani:2017mva}

Jet substructure has proven to be a powerful set of techniques for tagging blah blah~\cite{Abdesselam:2010pt,Altheimer:2012mn,Altheimer:2013yza,Adams:2015hiv}.  This study suggests blah blah.  This paper is organized as follows.  

Goal of jet substructure community

Application of new techniques beyond tagging

AIM:  Achieve something like 10\% precision to start to answer other questions

\subsection{Cite Dump}

PDG 

mMDT \cite{Dasgupta:2013ihk,Dasgupta:2013via}

filtering/pruning/trimming \cite{Butterworth:2008iy,Ellis:2009su,Ellis:2009me,Krohn:2009th}

Soft Drop \cite{Larkoski:2014wba}

Cambridge/Aachen \cite{Wobisch:1998wt,Dokshitzer:1997in}

Calculation 1 \cite{Frye:2016okc,Frye:2016aiz}

Calculation 2 \cite{Marzani:2017mva}

AKT \cite{Cacciari:2008gp}, FastJet \cite{Cacciari:2011ma}

Sudakov Safety \cite{Larkoski:2013paa,Larkoski:2015lea}, $z_g$ \cite{Larkoski:2015lea}

\end{comment}


