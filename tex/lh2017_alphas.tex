\documentclass[11pt,letterpaper]{article}
\pdfoutput=1
\usepackage{jheppub}

\usepackage[utf8]{inputenc}

\usepackage{color}
\usepackage{graphicx}
\usepackage{tabularx}
\usepackage{xspace}

\usepackage{verbatim}
\usepackage{amsmath}
\usepackage{amssymb}
\usepackage[caption=false]{subfig}
\usepackage{url}
\usepackage{bbold}
\usepackage{slashed}
\usepackage{array}
\usepackage{hepunits}

\usepackage{multirow}
\usepackage{threeparttable}
\usepackage{paralist}

%\newcommand{\GeV}{\text{GeV}}
%\newcommand{\TeV}{\text{TeV}}
\newcommand{\SO}{\text{SO}}
\newcommand{\SU}{\text{SU}}
\newcommand{\SM}{\text{SM}}

\newcommand{\U}{\text{U}}
\newcommand{\CKM}{\text{CKM}}
\newcommand{\eff}{\text{eff}}

\newcommand{\cO}{{\mathcal O}}

\newcommand{\genang}[2]{{\lambda^{#1}_{#2}}}

\newcommand{\lamqcd}{{\Lambda_\text{QCD}}}

\newcommand{\ev}{\text{event}}
\newcommand{\jet}{\text{jet}}
\newcommand{\jets}{\text{jets}}
\newcommand{\subj}{\text{subjet}}
\newcommand{\subjs}{\text{subjets}}
\newcommand{\cut}{\text{cut}}
\newcommand{\trim}{\text{trim}}
\newcommand{\Ecut}{E_{{\rm cut}}}

\newcommand{\ptc}{p_{T{\rm cut}}}
\newcommand{\ptsubc}{p_{T{\rm subcut}}}

\newcommand{\ecf}[2]{e_{#1}^{(#2)}} 
\newcommand{\ecfnobeta}[1]{e_{#1}} 

\newcommand{\sub}{\text{sub}}
\newcommand{\miss}{\text{miss}}

\newcommand{\pythia}{\textsc{Pythia~8}\xspace}
\newcommand{\herwig}{\textsc{Herwig~7}\xspace}
\newcommand{\eventtwo}{\textsc{Event2}\xspace}
\newcommand{\vincia}{\textsc{Vincia}\xspace}
\newcommand{\sherpa}{\textsc{Sherpa}\xspace}

\newcommand{\FastJet}{\textsc{FastJet}\xspace}
\newcommand{\MadGraph}{\textsc{MadGraph}\xspace}

\newcommand{\df}{\text{d}}
\newcommand{\vev}[1]{\langle #1 \rangle}


\DeclareRobustCommand{\Sec}[1]{Sec.~\ref{#1}}
\DeclareRobustCommand{\Secs}[2]{Secs.~\ref{#1} and \ref{#2}}
\DeclareRobustCommand{\Secss}[3]{Secs.~\ref{#1}, \ref{#2}, and \ref{#3}}
\DeclareRobustCommand{\App}[1]{App.~\ref{#1}}
\DeclareRobustCommand{\Tab}[1]{Table~\ref{#1}}
\DeclareRobustCommand{\Tabs}[2]{Tables~\ref{#1} and \ref{#2}}
\DeclareRobustCommand{\Fig}[1]{Fig.~\ref{#1}}
\DeclareRobustCommand{\Figs}[2]{Figs.~\ref{#1} and \ref{#2}}
\DeclareRobustCommand{\Figss}[3]{Figs.~\ref{#1}, \ref{#2}, and \ref{#3}}
\DeclareRobustCommand{\Eq}[1]{Eq.~(\ref{#1})}
\DeclareRobustCommand{\Eqs}[2]{Eqs.~(\ref{#1}) and (\ref{#2})}
\DeclareRobustCommand{\Eqss}[3]{Eqs.~(\ref{#1}), (\ref{#2}), and (\ref{#3})}
\DeclareRobustCommand{\Ref}[1]{Ref.~\cite{#1}}
\DeclareRobustCommand{\Refs}[1]{Refs.~\cite{#1}}

\newcommand{\be}{\begin{equation}}
\newcommand{\ee}{\end{equation}}
\newcommand{\nn}{\nonumber}

\renewcommand{\textfraction}{0.10}
\renewcommand{\topfraction}{0.90}
\renewcommand{\bottomfraction}{0.90}
\renewcommand{\floatpagefraction}{0.65}

%% Reference commands %%
\newcommand{\mb}[1]{\boldsymbol{#1}}
\newcommand{\bm}[1]{\boldsymbol{#1}}
\newcommand{\mbo}[1]{\boldsymbol{\overline{#1}}}

\usepackage{xspace}


\def\Tr{\mathop{\rm Tr}}
\newcommand{\rep}[1]{\mathbf{#1}}
\newcommand{\conjrep}[1]{\overline{\mathbf{#1}}}


\renewcommand{\a}{\alpha}
\renewcommand{\b}{\beta}
\newcommand{\e}{\epsilon}
\newcommand{\D}{\Delta}
\renewcommand{\l}{\lambda}
\renewcommand{\th}{\theta}
\newcommand{\bq}{\bar{q}}
\newcommand{\zcut}{z_{\rm cut}}
\newcommand{\ycut}{y_{\rm cut}}

\newcommand{\IZ}{\mathbb{Z}}
\newcommand{\cD}{\mathcal{D}}
\newcommand{\cL}{\mathcal{L}}
\newcommand{\cR}{\mathcal{R}}
\newcommand{\cF}{\mathcal{F}}
\newcommand{\cI}{\mathcal{I}}
\newcommand{\cK}{\mathcal{K}}
\newcommand{\beq}{\begin{eqnarray}}
\newcommand{\eeq}{\end{eqnarray}}

\newcommand{\F}{\mathcal{F}}
\newcommand{\Ft}{\widetilde{\mathcal{F}}}
\newcommand{\G}{\mathcal{G}}
\newcommand{\Gt}{\widetilde{\mathcal{G}}}
\newcommand{\HH}{\mathcal{H}}
\newcommand{\HHt}{\widetilde{\mathcal{H}}}
\newcommand{\ord}[1]{\mathcal{O}\!\left(#1\right)}

\newcommand*\numcircledmod[1]{#1 \!\!\! \bigcirc}

\newcommand{\Njet}{\widetilde{N}_{\rm jet}}
\newcommand{\dN}[1]{\Delta_{#1}}
\newcommand{\dNpm}{\Delta_{2\pm}}
\newcommand{\dNp}{\Delta_{2+}}
\newcommand{\dNm}{\Delta_{2-}}
\newcommand{\dNtm}{\Delta_{3-}}

\newcommand{\cT}{\mathcal{T}}
%\newcommand{\as}{\alpha_s}
\renewcommand{\angle}{\theta}

%\definecolor{darkgreen}{rgb}{0,0.5,0}
%\newcommand{\jdt}[1]{\textbf{\textcolor{darkgreen}{(#1 --jdt)}}}
%\newcommand{\gs}[1]{\textbf{\textcolor{darkblue}{(#1 --gs)}}}

\definecolor{mildred}{rgb}{0.8,0,0}
\newcommand{\info}[1]{\textbf{\textcolor{mildred}{(#1)}}}

\definecolor{llblue}{rgb}{0,0.5,1.0}
\newcommand{\ijm}[1]{\textbf{\textcolor{llblue}{(#1 --ijm)}}}

\definecolor{darkgreen}{rgb}{0,0.6,0.0}
\newcommand{\gs}[1]{\textbf{\textcolor{darkgreen}{(#1 --gs)}}}

\begin{document}


\title{Towards Extracting the Strong Coupling Constant \\ from Jet Substructure at the LHC}

%\author[a]{Grigorios Chachamis,}
\author[a]{Suman Chatterjee,}
\author[d]{Fr\'{e}d\'{e}ric Dreyer,}
\author[e]{Maria Vittoria Garzelli,}
\author[d]{Philippe Gras,}
\author[e]{Andrew Larkoski,}
%\author[g]{Daniel Maitre,}
\author[g]{Simone Marzani,}
\author[h]{Ian Moult,}
\emailAdd{ianmoult@lbl.gov}
\author[h]{Ben Nachman,}
\emailAdd{bpnachman@lbl.gov}
\author[i]{Andrzej Si\'{o}dmok,}
\author[j]{Andreas Papaefstathiou,}
\author[j,f]{Peter Richardson,}
\author[a]{Tousik Samui,}
\author[k]{Gregory Soyez,}
\emailAdd{gregory.soyez@ipht.fr}
\author[b]{and Jesse Thaler}
\emailAdd{jthaler@mit.edu}

%\affiliation[a]{Instituto de F\'{i}sica Te\'{o}rica UAM/CSIC \& Universidad Aut\'{o}noma de Madrid, Madrid, Spain}
\affiliation[a]{Tata Inst. of Fundamental Research, Mumbai, India}
\affiliation[b]{Center for Theoretical Physics, Massachusetts Institute of Technology, Cambridge, MA 02139, USA}
\affiliation[c]{II. Institute for Theoretical Physics, Hamburg University Luruper Chaussee 149, D-22761 Hamburg, Germany}
\affiliation[d]{IRFU, CEA, Universit\'{e} Paris-Saclay, Gif-sur-Yvette, France}
\affiliation[e]{Physics Department, Reed College, Portland, OR 97202, USA}
\affiliation[f]{IPPP, Department of Physics, Durham University}
\affiliation[g]{Dipartimento di Fisica, Universit\'{a} di Genova and INFN, Sezione di Genova, Via Dodecaneso 33, 16146, Italy}
\affiliation[h]{Physics Division, Lawrence Berkeley National Laboratory, 1 Cyclotron Rd, Berkeley, CA, 94720, USA}
\affiliation[i]{The Henryk Niewodniczanski Institute of Nuclear Physics in Cracow, Polish Academy of Sciences}
\affiliation[j]{Theoretical Physics Department, CERN, CH-1211 Geneva 23, Switzerland}
\affiliation[k]{IPhT, CEA Saclay, CNRS UMR 3681, F-91191 Gif-sur-Yvette, France}

\abstract{
Recent advances in jet substructure have introduced a new class of jet observables that are amenable to systematically improveable calculations in the complex environment of the Large Hadron Collider (LHC). These observables exploit grooming to reduce non-perturbative contributions, and simplify perturbative calculations. With these recent advances in both theory and experiment, we believe that it is the appropriate time to start to investigate the possibility of extracting the strong coupling constant $\alpha_s$ from jet substructure.  In this paper we perform a proof-of-principle sensitivity study to demonstrate the suitability of such measurements to add useful information to the existing precision extractions of $\alpha_s$. We highlight a number of theoretical and experimental advantages of groomed observables for $\alpha_s$ extractions, and discuss several difficulties of the LHC environment.   Using a simplified approach, we show that a measurement of $\alpha_s$ with an approximate 10\% uncertainty should be possible at the LHC.  This result motivates a complete analysis with a full set of experimental and theoretical considerations, and we present a number of directions where improvements on both the theory and experiment sides could be made in the near future.
}

\maketitle

\section{TODO}

\begin{itemize}
\item Cut off Fig.~\ref{fig:sensitivity} and Fig.~\ref{fig:robustness} at -12.  Also, add a statement or plot about Pythia.
\end{itemize}

\clearpage

%%==============
\section{Introduction}
%%==============

\info{Ben, Ian, Jesse, Grigorios, Maria}

Goal of jet substructure community

Application of new techniques beyond tagging

AIM:  Achieve something like 10\% precision to start to answer other questions

\subsection{Cite Dump}

PDG \cite{Olive:2016xmw}

mMDT \cite{Dasgupta:2013ihk,Dasgupta:2013via}

filtering/pruning/trimming \cite{Butterworth:2008iy,Ellis:2009su,Ellis:2009me,Krohn:2009th}

Soft Drop \cite{Larkoski:2014wba}

Cambridge/Aachen \cite{Wobisch:1998wt,Dokshitzer:1997in}

Calculation 1 \cite{Frye:2016okc,Frye:2016aiz}

Calculation 2 \cite{Marzani:2017mva}

AKT \cite{Cacciari:2008gp}, FastJet \cite{Cacciari:2011ma}

Boost Reports \cite{Abdesselam:2010pt,Altheimer:2012mn,Altheimer:2013yza,Adams:2015hiv}

Sudakov Safety \cite{Larkoski:2013paa,Larkoski:2015lea}, $z_g$ \cite{Larkoski:2015lea}

\begin{figure}
\begin{center}
\includegraphics[width = 0.6\columnwidth]{figures/alphas_propaganda.pdf}
\end{center}

\caption{Ben's propaganda plot}

\label{fig:propaganda}

\end{figure}

%%==============
\section{Observable and Algorithm Definitions}
\label{sec:definitions}
%%==============


In this section we briefly review the definitions of the observables and grooming procedures that we will focus on in this paper.


%%%%%%%%%%%%%%%%%%%%%%%%%%%%%%%%%%%%%%%
\subsection{Observables}\label{sec:shape_def}
%%%%%%%%%%%%%%%%%%%%%%%%%%%%%%%%%%%%%%%


The simplest class of jet shape observables that have sensitivity to the value of $\alpha_s$ are two-point correlation functions. There has been significant theoretical study of these objects, and their perturbative behavior is understood to relatively high accuracy. For a set of constituents $\{i\}$, in a jet $J$, the two-point correlation functions in $pp$ collisions are defined as

\begin{align}\label{eq:ppe2}
\left.\ecf{2}{\alpha}\right|_{pp}&=\frac{1}{p_{TJ}^2}\sum_{i<j\in J} p_{Ti} p_{Tj}\left(R_{ij}/R\right)^\alpha\,, 
\end{align}
where $R_{ij}$ is the distance between particles $i$
and $j$ in the rapidity-azimuthal angle plane.   The angular exponent $\alpha$ in the definition of the two-point correlation functions is a parameter that controls sensitivity to wide-angle emissions. For jets that are central and if all emissions in the jet are nearly collinear, then two-point energy correlation function reduces to a function of the jet mass ($m$) when $\alpha=2$:
\begin{equation}
\ecf{2}{2} \sim \frac{m^2}{R^2p_{T}^2}\,.
\end{equation} 

Throughout this paper, we will primarily restrict ourselves to the
case of $\alpha=2$ as this has been the focus so far from both the theory~\cite{Frye:2016okc,Frye:2016aiz,Marzani:2017kqd,Marzani:2017mva} and experimental~\cite{Aaboud:2017qwh,CMS-PAS-SMP-16-010} communities.  However, it is interesting
to study if other values of $\alpha$ could provide improved
sensitivity to $\alpha_s$. Another benchmark value is $\alpha=1$,
which corresponds to $k_T$ (or broadening) instead of mass.

%a broadening or $k_T$ type measure.

%%%%%%%%%%%%%%%%%%%%%%%%%%%%%%%%%%%%%%%
\subsection{Grooming Techniques}\label{sec:groom_tech}
%%%%%%%%%%%%%%%%%%%%%%%%%%%%%%%%%%%%%%%

Starting from a jet identified with an IRC safe jet algorithm (such as
anti-$k_t$ \cite{Cacciari:2008gp}), the soft drop algorithm is defined
using Cambridge/Aachen (C/A) reclustering
\cite{Dokshitzer:1997in,Wobisch:1998wt,Wobisch:2000dk}.  \begin{enumerate}

\item Recluster the jet using the C/A clustering algorithm, producing an angular-ordered branching history for the jet.

\item Step through the branching history of the reclustered jet.  At each step, check the soft drop condition
\begin{align}\label{eq:sd_cut}
\frac{\min\left[ p_{Ti}, p_{Tj}  \right]}{p_{Ti}+p_{Tj}}> \zcut \left(   \frac{R_{ij}}{R}\right)^\beta \,.
\end{align}
where $\zcut$ is a parameter defining the scale below which soft radiation is removed.  If the soft drop condition is not satisfied, then the softer of the two branches is removed from the jet.  This process is then iterated on the harder branch.

\item The soft drop procedure terminates once the soft drop condition is satisfied.

\end{enumerate}

\noindent This procedure generalizes the modified Mass Drop
  Tagger~\cite{Dasgupta:2013ihk} which corresponds to $\beta=0$ in the
  Soft Drop procedure described above.  Note that non-global logarithms are formally removed from the soft drop mass distribution, even when $\beta>0$ (though clearly they are resorted as $\beta\rightarrow\infty$).

%Specializingto the case of $\beta=0$, where Soft Drop is equivalent to Modified Mass Drop Tagger~\cite{Dasgupta:2013ihk}, the algorithm proceeds as follows: \gs{The descriptino below is valid for any $\beta$, so I'd just drop the $\beta=0$ part here and mention, below the description ``This generalises the modified Mass Drop Tagger~\cite{Dasgupta:2013ihk} which corresonds to $\beta=0$ in the Soft Drop procedure described above.''}

%For $\beta=0$ non-global logarithms are removed from the distribution.  \gs{This is a bit misleading: NGLs are also formally removed from Soft  Drop with $\beta>0$}






%%==============
\section{Grooming Away Soft Complications}
\label{sec:softcomplications}
%%==============

%\info{Andrew, Simone}


Generic result of grooming is removing wide-angle soft radiation.  What impact does this have?

\subsection{Mitigating Nonperturbative Effects ($e^+e^-$ and $pp$)}

\info{IAN}

\begin{itemize}
\item grooming original purpose: reduce sensitivity to soft-wide angle radiation
\item it obviously reduces the impact of UE and pile-up in pp collision
\item what about hadronization? at first sight less obvious because we get rid of soft radiation but we also reduce the effective radius. Competing effects?
\item however it helps with hadronization too: parametric understanding (here there is some calculation in mMDT paper and Harvard paper too). Some unpublished studies exist on  for $\beta>0$ by Gregory, Lais and SM. 
\item Abundant MC evidence that helps
\item Open question: any deeper understanding in terms of shape functions?
\end{itemize}

Comparison to shape function in thrust.  Grooming gives different sensitivity to NP effects.

In standard thrust fits, whole distribution shifts from NP.  Degenerate with $\alpha_s$ shift, hard to disentangle.

Grooming pushes NP effects to separate region.  Separation of NP region, resummation region, fixed order regime (at high enough jet $p_T$).

\subsection{Process Independence at Hadron Colliders ($pp$)}
\info{IAN}

clearly there still is some process-dependence, in terms of q/g fractions as well as hard coefficients. It is however much reduced.
compare to issue of PDF, 3-jet over 2-jet, ECF extractions
Grooming removes soft correlations: it ``turns the LHC into an e$^+$e$^-$ machine (too strong?)
Related:  Grooming remove NGLs and other contamination. It makes things easier to calculate. 


\subsection{Improved Detector Resolution (LHC)}
\info{BEN}

Cites to pileup study. The jet mass sensitivity to pileup is $\mathcal{O}(A^4)$~\cite{Salam:2009jx}.  There are constituent-based pileup mitigation techniques~\cite{Cacciari:2014gra,Krohn:2013lba,Bertolini:2014bba,Berta:2014eza,Komiske:2017ubm}, but ...

Grooming needed to help pileup from LHC


%%==============
\section{Observable Sensitivity to $\alpha_s$}
\label{sec:jetmass}
%%==============

%\info{Andrew, Simone}

Soft drop mass is most accurate jet substructure observable to date.  Review soft drop, and current calculations

\subsection{Soft Drop Declustering}

Review of soft drop


The Soft Drop grooming procedure~\cite{Larkoski:2014wba} takes a jet
with momentum $p_t$ and radius $R$. It re-clusters its constituents
using the Cambridge/Aachen (C/A) algorithm \cite{Dokshitzer:1997in,
  Wobisch:1998wt} and iteratively performs the following steps:
\begin{enumerate}
 \item it de-clusters the jet into 2 subjets $j \to j_1 + j_2$;
 \item it checks the condition 
\begin{equation}\label{eq:sd-condition}
\frac{\min (p_{t1} , p_{t2})}{p_{t1}+p_{t2}} > \zcut \left(
  \frac{\theta_{12}}{R}\right)^\beta\,;
\end{equation}
\item if the jet passes the condition, the recursion stops; if not the
  softer subjet is removed and the algorithms goes back to step 1 with
  the hardest of the two subjets. 
 \end{enumerate}
In the case $\beta=0$ Soft Drop essentially reduces to mMDT~\cite{Dasgupta:2013ihk},
 albeit without any actual mass-drop condition. Moreover, while the
 original MDT~\cite{Butterworth:2008iy} algorithm imposed a cut on the ratio of angular distances
 to masses, a so-called $\ycut$, the mMDT variant instead cuts on
 momentum fractions~\cite{Dasgupta:2013ihk} (see
 e.g. \cite{Dasgupta:2013ihk,Dasgupta:2016ktv} for a comparison
 between $\ycut$ and $\zcut$).

\subsection{Method 1 (name?) Analytic Calculation}

Andrew, et al.

\subsection{Method 2 (name?)  Analytic Calculation}

Simone and Gregory, et al.

Do we want a comparison plot of the two calculations?
(calculated at NNLL, good enough to start extraction)

\subsection{Concrete Benefits of Grooming}

Show that grooming does what's advertised in \Sec{sec:softcomplications}.
Reduced sensitivity to NP effects but also better agreement with MC? 

\clearpage

%%==============
\section{Idealized Performance at the LHC}
\label{sec:ben_study}
%%==============

%\info{Ben, Andrzej, Jesse, Gregory, Grigorios, Frederic}

\subsection{Extraction of Theory Templates}

	Theory Templates.

	Ben is doing this.

	Current best NNLO + NLO for dijets, but for time, drop both first Ns
 
 Can do NLO + LO on days timescale (even if theory uncertainty is too large)

 Right now, assume known quark/gluon function, relax later.


\subsection{Estimate of Experimental Resolution}

	Parametrized Experimental Resolution?

	Ben has a toy simulation
	Resolution achievable, uncertainty?
	How to we think this will be done in this study (can we use fast simulation, or parametrization?)
	Statistical uncertainties at high pT?
	Pileup:  there are studies of this with current level of pileup, not a big deal yet

	Uncertainty worse at low and high mass, need to derive parametrized uncertainty
	
	Trade off with pT (rate versus NP control)

\begin{figure}
\begin{center}
\includegraphics[width = 0.49\columnwidth]{figures/experimentaldemo/SD_resolution.pdf}\includegraphics[width = 0.49\columnwidth]{figures/experimentaldemo/resolution_scan.pdf}
\end{center}
\caption{blah blah}
\label{fig:expfit}
\end{figure}

\subsection{Fit in Pure Quark/Gluon Samples}

Fit Methodology:
	Which fix range (minimizing theory and experimental uncertainties)?
	Constraining quark vs gluon fraction (varying SD parameters)?
	Zeroth-order feasibility study
	
	Plot is for distribution folded over PDFs (can we get rid of that?)

	Choice of jet radius (varying resummation scale)

Assume (for this section) limited by experimental uncertainties



\subsection{Constraining Quark/Gluon Fraction with Data}

	Question:  constrain quark/gluon fraction (adjust $z_cut$)?
	Need to decide beta and zcut values
	Need to matching to fixed order
	beta = 0 mass is baseline
	
	
	Another study to mitigate quark/gluon fraction uncertainties
	Sensitivitity to PDF only to the extent of getting quark/gluon fraction
	Can mitigate that with fit.




%%==============
%\section{Assessing Theoretical Uncertainties}
%%==============

%\info{Simone, Andrew, Frank, Ian, Gregory, Frederic}

Above analysis was with no (or small uncertainties).  What are leading uncertainties?  Estimate.

\subsection{Perturbative Scale Variations}

	Understanding scales at which $\alpha_s$ is being probed (as a function of zcut and beta)
		3-jet / 2-jet probes TeV
		But jet shapes probe lower scale than scale of the jet (which scale)?

	Perturbative Scale Variation (NLL/NNLL)


\subsection{Nonperturbative Effects from Shape Function}

	Treat NP effects (shape function?)  	Do we need shape function (or just a cut value)?

	Need to think about $\Omega_0$ issue (is it a shape-function-like shift?  Or just parametrically correct?)


\subsection{Finite $z_{\rm cut}$ Corrections}

	Finite zcut effects.  resummation vs.  power corrections?

\subsection{Fixed-Order Corrections}

	Matching to fixed-order?
		Universality in resummation region...
		...but process-dependence in fixed order region
		Efficiency of matching using NLO, making grid is painful
		Get help from FOMC expert
		LO matching is tree-level, e.g. MadGraph, PDFs...
		Matching to Fixed Order (remove the region from the fit, or figure out efficient calculations strategy.)

	Does peak give $\alpha_s$ information?
	
\subsection{Sensitivity to Initial State Effects }

	Parton Distribution Functions, 	Initial State Radiation, Underlying event.




%%==============
%\section{Alternative Observables}
%%==============

%\info{Gregory, Jesse, Ian}

Beyond Soft Drop Mass?

\subsection{Additional Soft Drop Observables}

Angularities and $R_g$.

\subsection{Single-emission Observables}

Possible Dump this

\subsection{Track-based Observables}

Insensitivity to pileup, improved angular resolution.
	Tracking observables more sensitive




%%==============
%\section{Estimated Sensitivity from Parton Showers}
%%==============

%
\info{Andrzej, Johannes, Maria, Frederic, Peter, Tousik}

Estimated Sensitivity in Parton Shower MC

Adjust $\alpha_s$ in the final state PS and see the fit (make sure you see the same thing)

\subsection{Samples}

	Use pure Z + quark, pure Z+ gluon, and  dijet
	
	MadGraph hard scattering, interfaced with Pythia/Herwig
	
	13 TeV
	PDF choice?  (default per generator, or from MadGraph)
	Parton-level in MadGraph Born-level ($p_t > 450~\GeV$)
	Start with 100k events.
	
	
	$pT > 500~\GeV$, anti-$k_t$ $R = 0.8$, $|eta| < 2.5$

	
	
	Measure Pythia $\alpha_s$  with Herwig?
	
	Observables in RIVET

\subsection{Observables}


Observables:
-- Baseline:  mMDT mass (zcut = 0.1, beta = 0)
-- Sweeps:  zcut = 0.05, 0.1, 0.2
-- beta = 0, 1, 2
-- Angularity:  alpha = 0.5, 1.0, 2.0
-- $R_g$
-- Plus track based
-- No single emission for this study

\subsection{Extracting Parton Shower Templates}

	Use templates, e.g. $\theta_g$ distirbution as a function $\alpha_s$


\subsection{Consistency Check for Soft Drop Mass}
	Compare to analytic or parton shower study (do a closure test)

	Try different ME matching prescriptions, see if they close

	Plot of $\alpha_s$ versus log likelihood (statistical ssensitivity)

\subsection{Test of Alternative Observables}

Make sure that you measuring something correlated with $\alpha_s$




%%==============
%\section{Prospects at Lepton Colliders}
%%==============

%

\info{Jesse}

	Look at e+e- case of jet shapes there (no quark fraction issue)

	Make a statement on e+e- (LEP) as well (can it do better than thrust?)
		Probes lower scales, so still interesting

	Limit our scope to e+e- is an option, though could just be statement, no study

	If we do e+e-, has to be compared to thrust



\clearpage

%%==============
\section{Conclusions and Future Outlook}
%%==============
\label{sec:future}
%\info{All}

In this paper we have performed a preliminary study of the possibility of performing an extraction of the strong coupling constant $\alpha_s$ from groomed jet substructure measurements at the LHC.  This has been made possible by recent advances in the calculation of groomed event shapes, which allow them to be resummed to NNLL accuracy, as well as advances in the understanding of the experimental uncertainties for groomed jet observables. 

We have highlighted a number of features that an extraction of
$\alpha_s$ from groomed substructure would offer. In particular, the
grooming procedure suppresses non-perturbative hadronization
corrections to the mass distribution extending the range of
perturbative control by up to a factor of $\sim 100$ as compared to the ungroomed case, where hadronization corrections are a dominant uncertainty. Furthermore, the nature of the non-perturbative corrections is completely different, and therefore a groomed extraction could provide complementary information. 

Using the groomed jet mass as a concrete example, we have performed a feasibility study for the extraction of $\alpha_s$ from groomed jet substructure. Taken separately, the groomed quark and gluon distributions both exhibit good sensitivity to $\alpha_s$. However, this study also highlighted a key issue relevant at the LHC. The leading behavior of the shape of the distribution is sensitive to the product $\alpha_s C_i$, where $C_i$ is the color Casimir, namely $C_A$ for gluon jets and $C_F$ for quark jets. This immediately implies that there is a degeneracy between the value of $\alpha_s$ and the quark gluon fraction of the sample. We highlighted two ways of overcoming this, either a fixed order calculation of the quark and gluon fractions, or the measurement of different observables, which enables a breaking of this degeneracy. This would ideally be combined with multiple samples with (relatively pure~\cite{Gallicchio:2011xc}) quark/gluon fractions.  Significantly better precision is achieved by the explicit calculation, however, this also introduces a stronger dependence on the PDFs which should then ideally be simultaneously fitted.

With currently achievable experimental and theoretical uncertainties, we have shown that an extraction of $\alpha_s$ at the $10\%$ level is feasible using the currently available data at the LHC.  We believe that this study motivates a serious effort to extract the strong coupling constant, $\alpha_s$ from jet substructure at the LHC using groomed jet shapes. A serious analysis will require advances on both the theory and experiment side. We therefore conclude by discussing some of the major obstacles that must be overcome.

%\ijm{Ben will say something about experiment}


On the theory side, we can separately discuss the three primary aspects of the calculation:

\vspace{5mm}

\noindent {\bf Resummation Accuracy:} For $e^+e^-$ event shapes, the current state of the art is N$^3$LL. This has been achieved only for a few select observables using the soft collinear effective theory. The extension  to N$^3$LL accuracy for the groomed jet mass would require only the calculation of the anomalous dimensions for the groomed soft function to three-loop. This could be performed using currently available techniques, and would enable N$^3$LL accuracy. 

At this level of accuracy it will also be important to asses perturbative power corrections in $\zcut$.  Reference~\cite{Marzani:2017kqd,Marzani:2017mva} has shown that these corrections are numerically small.  However, if multiple $z_\text{cut}$ values are used that are not much less than one, these may be important.

\vspace{5mm}

\noindent {\bf Fixed Order Matching:} A key aspect, and potential complication to achieve a competitive theoretical accuracy as for $e^+e^-$ is the fixed order matching. For $e^+e^-$  the perturbative corrections to $e^+e^- \to$ 3 jets are known to NNLO \cite{GehrmannDeRidder:2007hr,Gehrmann-DeRidder:2007nzq,Weinzierl:2008iv,Weinzierl:2009ms}. To achieve a similar perturbative accuracy for the matching for the groomed jet mass will require $2\to 3$ matrix elements at NNLO. While results for the amplitudes are just becoming available \cite{Gehrmann:2015bfy,Dunbar:2016aux,Badger:2013yda,Badger:2017jhb,Abreu:2017hqn}, it will be a while before numerically efficient evaluations of the relevant cross sections are available. We believe that this is currently the theoretically most difficult ingredient. 

\vspace{5mm}

\noindent {\bf Non-Perturbative Corrections:} Finally, in addition to improving our understanding of the perturbative accuracy of the observable, it will also be crucial to improve our understanding of the non-perturbative aspects of groomed observables. While it has been shown through Monte Carlo studies that non-perturbative effects are suppressed throughout a large component of the distribution, it will be important to quantify this further. In the case of $e^+e^-$ event shapes, operator definitions of soft matrix elements allow for field theoretic definitions of the non-perturbative parameters. Ideally this could also be performed for groomed observables, placing the groomed jet mass on a firmer theoretical footing.

\vspace{5mm}

Experimentally, there are three broad topics that need to be addressed for a precision extraction of $\alpha_s$:

\vspace{5mm}

\noindent {\bf Mass Scale Uncertainties:} At the moment, ATLAS~\cite{Aaboud:2017qwh} and CMS~\cite{CMS-PAS-SMP-16-010} have very different approaches for determining the jet mass scale uncertainty, both with known limitations.  CMS performs a fit to the hadronic $W$ boson mass peak in one-lepton $t\bar{t}$ events and takes the shift in the peak position as the uncertainty in the jet mass scale (which is negligible and thus ignored)~\cite{Sirunyan:2016cao}.  Two challenges with this approach are that (a) the peak position is a convolution of particle-level and detector-level effects and (b) it is not clear that uncertainties derived for boosted $W$ bosons should be the same as for generic quark and gluon jets at all masses.  One can overcome (a) with a technique like forward-folding~\cite{ATLAS-CONF-2016-008,ATLAS-CONF-2016-035}.  Various PS MC generators are studied, but it is likely not sufficient to have one global model comparison.  The impact on the jet mass resolution in one topology may be completely different than the impact of the scale, resolution, or acceptance in another topology.  In contrast, the ATLAS measurement propagates constituent-based uncertainties through to the groomed mass.  These uncertainties are derived from matching tracks to calorimeter-cell clusters and studying the energy and angular matching.  Studies have shown that this `bottom-up' approach works well for reproducing the jet energy scale~\cite{Aaboud:2016hwh}, which has been validated also for groomed jets in Ref.~\cite{Aaboud:2017qwh}.  However, this does not hold exactly for the mass, which is not linear in the constituent energies~\cite{Nachman:2016qyc}.   The uncertainties are validated using the standard ATLAS approach using track-jets~\cite{Aad:2013gja,ATLAS-CONF-2017-063}, but to achieve higher precision, a more detailed understanding of the impact of energy thresholds, fluctuation correlations and calorimeter cluster merging will be required.

\vspace{5mm}

\noindent {\bf Mass Resolution Uncertainties:} ATLAS and CMS use the same approaches for the resolution as for the scale (bottom-up and $W$ mass peak).  ATLAS validates their approach in a similar manner as CMS, by using the $W$ mass peak from $t\bar{t}$ events.\footnote{This validation does not have the same complete forward-folding machinery as was used by ATLAS for trimmed jets.}  The mass resolution is mostly not the limiting factor, but it could be if the situation is not improved as the jet mass scale precision improves. 

\vspace{5mm}

\noindent {\bf Pileup Modeling and Mitigation:} Grooming significantly reduces the impact of pileup, but if the Run 3+ data are to be used (pileup levels of 80+), then a significantly better and more detailed understanding of the degradation due to pileup will be required.  Statistical are currently not dominant, so it is conceivable that the higher instantaneous luminosity data is not used for the precision $\alpha_s$ extraction.  This may change if one wants to exploit the largest lever-arm possible; to access the highest $p_\text{T}$ jets, we will need more data.

\vspace{5mm}

It would also be interesting to study in more detail the use of other groomed observables. Since it may be the case that it is ultimately experimental uncertainties that are the limiting factor, one potentially interesting direction is the use of track based measurements. From the experimental perspective, this significantly reduces uncertainties, particularly in a high pile up environment. From the theoretical side, it introduces the need for non-perturbative track functions. However, for specific observables, only certain moments of the track functions are required. This therefore offers the potential that combined fits for the moments of the track functions and $\alpha_s$ could be performed, much like how fits are performed for $e^+e^-$ event shapes. Significantly more theoretical work is required to see if this is truly a viable possibility, although we believe that this is well motivated.


We are optimistic that these difficulties on both the theory and experiment side can be overcome, enabling a precision probe of the strong coupling constant from jet substructure in the LHC environment.

\begin{acknowledgments}

The work of GS is supported in part by the French Agence Nationale de la Recherche,
under grant ANR-15-CE31-0016, and by the ERC Advanced Grant Higgs@LHC
(No.\ 321133).
%
The work of JT is supported by the DOE under grant contract numbers DE-SC-00012567 and DE-SC-00015476.
%
IM and BN are supported in part by the Office of High Energy Physics of the U.S. Department of Energy under Contract No. DE-AC02-05CH11231, and the LDRD Program of LBNL.

\end{acknowledgments}

\bibliographystyle{jhep}
\bibliography{lh2017_alphas}

\end{document}
